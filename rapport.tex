\documentclass[12pt]{article}
\usepackage[utf8]{inputenc}
\usepackage[T1]{fontenc}
\usepackage[francais]{babel}
\usepackage{hyperref}
\author{LABBE Emeric, PINERO Alexandre}
\title{Rapport de projet d'Architecture des Ordinateurs}

\begin{document}
\maketitle
\tableofcontents

\newpage
\section{Exercice 1 : De la place dans les opcodes}
\subsection{Factorisation de iaddl etc. avec addl etc.}
Dans cette partie, pour l'implementation sequentielle il y avait un piège, c'était de changer la taille des instructions contenues dans OPL et RRMOVL. Donc dans le fichier "misc/isa.c", nous avons passé la taille des instructions \emph{addl, subl, andl, xorl, sall et sarl} à 6 au lieu de 2 ainsi elles ont désormais la même taille que les instructions \emph{iaddl, isubl, iandl, ixorl, isall et isarl}.\par
\`A l'étage Execute, les IOPL utilisent valC et les OPL valA. Pour les différentier, on regarde si ra est égal à 8 (ou RNONE) si c'est le cas on passe valC à l'ALUA sinon valA.

\subsection{Factorisation de irmovl avec rrmovl}
La factorisation des deux commande est la même chose que pour les OPL et les IOPL, il faut passé la taille de RRMOVL à 6 et utiliser valC dans l'ALUA si ra est égal à RNONE si ce n'est pas le cas utiliser valA.

\newpage
\section{Exercice 2 : Ajout du support d'unstruction sur plusieurs cycles}
\subsection{Version séquentielle}
Ce n'était pas la chose la plus compliqué à faire, mais il fallait la faire correctement. Dans le fichier "seq/std-seq.hcl" nous avons ajouté l'instructions :
\begin{verbatim}
	int instr_next_ifun = [
		1 : -1;
	];
\end{verbatim}
Ce bloc indique au processeur quelle valeur de ifun utiliser après celle que l’on vient de traiter.
Par convention, la valeur -1 indique que l’instruction est finie, et qu’il faut passer à la suivante.
Le reste du code se trouve dans le fichier "seq/ssim.c"
\subsection{version Pipe-linée}

\newpage
\section{Exercice 3 : Ajout d'instructions}
\subsection{L'instruction "enter"}
Pour ajouter l'instruction enter aux outils Y86 il a fallu modifier 5 de nos fichiers.
Tout d'abbord dans isa.h nous avons ajouté l'opcode I\_ENTER dans le itype\_t.

Dans yas-grammar.lex nous avons ajouté enter à la liste des instructions.

Dans isa.c nous avons ajouté enter et enter1 au tableau instruction\_set avec l'instruction suivante :
\begin{verbatim}
    {"enter",   HPACK(I_ENTER, 0), 1, NO_ARG, 0, 0, NO_ARG, 0, 0 },

\end{verbatim}

On lui donne ici la nouvelle instruction enter, qui est la première instruction à utiliser I\_ENTER. Le paramètre 1 correspond à la taille de l'instruction.

Enfin nous avons ajouté intsig ENTER dans les déclarations des fichiers seq-std.hcl et pipe-std.hcl.

Ensuite nous avons implémenté cette instruction dans ces deux fichiers avec 2 comportements différents (ifun = 0 et ifun = 1).

\subsubsection{Fetch Stage}
Nous avons ajouté ENTER à la liste instr\_valid, puis nous avons mis la condition : 
\begin{verbatim}
icode == ENTER && ifun == 0 : 1;
\end{verbatim}

Ainsi après le premier passage de ENTER, le second prendra le meme icode et le ifun suivant.

\subsubsection{Decode Stage}
Quand on a icode == ENTER : srcA = REBP et srcB = RESP.

Quand on a icode == ENTER et ifun == 1 : dstE = REBP (instruction rrmovl, \%esp, \%ebp). Sinon dstE = RESP(instruction pushl \%ebp).

\subsubsection{Execute Stage}
Dans le cas où ifun == 1 : on soustrait 4 à valB :
\begin{verbatim}
aluA : icode == ENTER && ifun == 0 : -4;
aluB : icode == ENTER : valB;
\end{verbatim}

Dans le cas où ifun == 0 : valB ne change pas :
\begin{verbatim}
aluA : icode == ENTER && ifun == 1 : 0;
aluB : icode == ENTER : valB;
\end{verbatim}

\subsubsection{Memory Stage}
On ajoute a mem\_write l'instruction ENTER avec :
\begin{verbatim}
icode == ENTER && ifun == 0
\end{verbatim}

On écrit ensuite le contenu de valA à l'adresse valE :
\begin{verbatim}
mem_addr : icode == ENTER && ifun == 0 : valE;
mem_data : icode == ENTER && ifun == 0 : valA;
\end{verbatim}

\subsection{L'instruction "mul"}

Tout comme l'instruction "enter", nous avons ajouté l'opcode I\_MUL dans le fichier "mics/isa.h", nous l'avons ajouté à la liste des instructions dans yas-grammar.lex et ajouter trois entrées dans le tableau des instructions de "mics/isa.c". Nous avons choisi de l'implementer de la façon suivante : 
\begin{verbatim}
EAX = 0;
while(ECX != 0){
    ECX--;
    EAX += EBX;
}
\end{verbatim}

Les trois entrées dans le tableau servent alors aux trois étapes de l'algorithme :

\begin{itemize}
	\item 0 - on initialise EAX
	\item 1 - on décrémente ECX
	\item 2 - on ajoute EBX à EAX si ECX supérieure à 0
	\item -1 - on sors de la boucle
\end{itemize}

Ainsi dans le code hcl de la version sequentielle, on commence avec l'ifun à 0 
pour initialiser \%eax à 0 puis on change d'état entre l'ifun à 1 et à 2. On fait cette opétation tanque la variable cc est différente de 2.
Autrement l'ifun passe à -1 et on passe à l'instruction suivante.

\subsection{L'instruction "repstos"}

\end{document}


